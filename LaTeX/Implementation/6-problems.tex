\section{Issues}

%In addition to listing problems and issues with the document or with the implementation process, this is also a spot to reflect upon the project and discuss what you would have done differently if you were to start again from scratch.

It turned out that the Skiller framework was poorly documented and it gave some unreadable exception messages. This slowed down the testing quite a bit. In addition, because we normally have only had two Android devices at our disposal, much of the testing have been done in the emulator. This is of course not very effective. Also, when running the application in the emulator and creating games, players would automatically join without user interaction. The Skiller team was unable to give us any good answers to why we experienced this problem.\\

We underestimated the importance of groupwide understanding of the framework, and the implementation should have been done collectively. As it was, we assigned one person to this task, and the rest of the group were unable to do anything with the framework in his absence. We also should  have a done a more thourogh research regarding the use of the framework and its documentation. \\

We have encountered an error while running the application, caused by an empty message from the Skiller server. This has been logged at runtime, but we have not been successful in figuring out if it was a local error or if it came from the server side. This error makes the game stop responding, and is hard to trace due to the fact that it is a periodic failure. \\

\subsection{Gained experieces}
In future projects we will spend more time researching possibilities when choosing to work with a third party framework. We did not look into the documentation of the framework before implementing it, and if we had, we would probably have chosen a different framework. We will also perform a quick search for problems related to it before making a final decision.  \\

Some members of the group went into this project with more experience than others, developing native Android applications. Due to good communication we have been able to use this experience to our advantage. \\

We have gotten a better understanding of the patterns we chose to implement. The experience of implementing a MVC pattern in a native Android application has been very useful. The state pattern and its usage was unknown for all of us, but we now have a good understanding of how it can be used. We could perhaps made this pattern a bit more clear, and assign more responsibilities to the different states.