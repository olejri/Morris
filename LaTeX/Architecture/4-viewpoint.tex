\newpage
\clearpage

\section{Architectural viewpoint selection}

The 4+1 View Model consist of four different views: Logical View, Development View, process View and physical View. Our game will be a 2-player network game, but since we are going to use a third party library for the network communication we will not use the Physical View.

\subsection{Logical view}

\begin{table}[h!]
\begin{tabular}{ | p{90pt} | p{270pt}  |}
\hline
Basis	 & We will put the system into perspective and get an overview of the software architecture. Standard procedure is to divide the program into different object-oriented models and find their relationship.  \\ \hline
Stakeholders & Developers, course teachers\\ \hline 
Description & Standard Class diagram with package notation \\ \hline

\end{tabular}

\caption{Logical view}

\end{table}

\vspace{-20pt}

\subsection{Development view}

\begin{table}[h!]
\begin{tabular}{ | p{90pt} | p{270pt}  |}
\hline
Basis	 & Used to get a perspective view of the different main blocks of the systems. \\ \hline
Stakeholders & Course teachers, Developers\\ \hline 
Description & Layer diagram \\ \hline

\end{tabular}

\caption{Development view}

\end{table}

\vspace{-20pt}

\subsection{Process view}

\begin{table}[h!]
\begin{tabular}{ | p{90pt} | p{270pt}  |}
\hline
Basis	 & We use the process view to get an overview of the process flow. This can help the developers to design and understand the main logic and structure of the game. \\ \hline
Stakeholders & Developers, Course teachers\\ \hline 
Description & Activity/Sequence diagram \\ \hline

%\addcontentsline{lot}{table}{Process view}

\end{tabular}

\caption{Process view}

\end{table}

\vspace{-20pt}

\subsection{Physical view}

\begin{table}[h!]
\begin{tabular}{ | p{90pt} | p{270pt}  |}
\hline
Basis	 & We use the physical view to get an on overview of software components on the physical layer, and communication between these components  \\ \hline
Stakeholders & Developers, Course teachers\\ \hline 
Description & Deployment diagram \\ \hline

%\addcontentsline{lot}{table}{Process view}

\end{tabular}

\caption{Pysical view}

\end{table}

\vspace{-20pt}







