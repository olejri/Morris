\pagebreak

\section{Views}
\subsection{Logical view}
\label{section:logical_view}

\begin{figure}[H]
%\vspace{-30pt}
\begin{center}
\includegraphics[width=1.2\textwidth]{./Images/LogicalView_final.png}
\end{center}
\caption{Logical view}
\end{figure}

The diagram follows the 4+1 logic view notation suggested by the Kructhen article \cite{krutchen}. The class diagram shows the structure of a system by showing the system's classes  and the relationships among them. Aggregation and inheritance is also displayed.

\subsection{Development view}

\begin{figure}[H]
\begin{center}
\includegraphics[width=220pt]{./Images/DevelopmentView.png}
\end{center}
\caption{Development view}
\end{figure}

The development view shows how we have organized our project. The diagram shows a rough overview of how the packages are organized. We have separated our models, views and activities (within Core) in different packages for a better overview.

\pagebreak

\subsection{Process view}

\begin{figure}[H]
%\vspace{-30pt}
\begin{center}
\includegraphics[width=200pt]{./Images/ProcessView}
\end{center}
\caption{Process view}
\end{figure}

The game is a turn based game with two players. In the activity diagram the progress of a game is described. When a game is started one of the players starts placing one of his pieces. The game checks if the player get three in a row. In that case the player can remove one of the opponents pieces. If the opponents have less then three pieces left, the other player have won. The diagram shows how a game round will pan out. 

\subsection{Physical view}

\begin{figure}[H]
%\vspace{-30pt}
\begin{center}
\includegraphics[width=200pt]{./Images/PhysicalLayer1}
\end{center}
\caption{Physical view}
\end{figure}

The physical view shows how the clients interact with the server of the Skiller framework. This is included for visualization only.






