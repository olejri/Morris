\newpage

\section{Quality requirements}

\subsection{Modifiability}

\subsubsection{M1 - Add new game variant}

Extending the game by adding an additional game board, where the player starts with 12 pieces instead of 9. To comply  with the setup of \emph{Twelve Men's Morris}. Additionally the board get four new paths the pieces can move in, to take the increase of pieces in consideration.

\begin{table}[h!]
\begin{tabular}{ | p{110pt} | p{250pt}  |}
\hline
\bf Source of stimulus &  Developer  \\ \hline
\bf Stimulus & Adding a new game variants \\ \hline 
\bf Artifact & A change that operates with target system  \\  \hline
\bf Environment & At design time \\ \hline
\bf Response & Extends functionality \\ \hline
\bf Response Measure & 8 hours to complete the modification \\ \hline

\end{tabular}
\caption{M1 - Add new game variant}
\end{table}

\pagebreak

\subsubsection{M2 - Computer opponent}

Extending the game by making it possible to play single-player against the computer, whereas the computer has three different skill levels ranging from easy to hard. Easy will be random placement of the pieces, medium will require some skill to beat and hard will at worst draw against a human.

\begin{table}[h!]
\begin{tabular}{ | p{110pt} | p{250pt}  |}
\hline
\bf Source of stimulus &  Developer  \\ \hline
\bf Stimulus & Adding a computer based player \\ \hline 
\bf Artifact & A change that operates with target system  \\  \hline
\bf Environment & At design time \\ \hline
\bf Response & Extends functionality \\ \hline
\bf Response Measure & 15 hours to do the modification \\ \hline

\end{tabular}
\caption{M2 - Computer opponent}
\end{table}

\subsubsection{M3 - Chat}

Will add a new dimension to the game, where people can chat to each other discussing tactics revolving around the game or just to socialize. The chat room is created together with the game, so you will only be able to chat with those you have an active game with.

\begin{table}[h!]
\begin{tabular}{ | p{110pt} | p{250pt}  |}
\hline
\bf Source of stimulus &  Developer  \\ \hline
\bf Stimulus & Adding game chat \\ \hline 
\bf Artifact & A change that operates with target system  \\  \hline
\bf Environment & At design time \\ \hline
\bf Response & Extends functionality \\ \hline
\bf Response Measure & 10 hours to do the modification \\ \hline

\end{tabular}
\caption{M3 - Chat}
\end{table}

\pagebreak

\subsubsection{M4 - Time constraint}

Adding functionality where a player has a limited time limit to decide his move, the time limit is set before the game starts. This can make it a more fast-paced game for those who want a more action-filled experience, and can lead to less frustrating waiting time.

\begin{table}[h!]
\begin{tabular}{ | p{110pt} | p{250pt}  |}
\hline
\bf Source of stimulus &  Developer  \\ \hline
\bf Stimulus & Adding time constraints \\ \hline 
\bf Artifact & A change that operates with target system  \\  \hline
\bf Environment & At design time \\ \hline
\bf Response & Extends functionality \\ \hline
\bf Response Measure & 3 hours to do the modification \\ \hline

\end{tabular}
\caption{M4 - Time constraint}
\end{table}

\subsection{Usability}

\subsubsection{U1 - Learning game rules}

If the user don't know the game rules, he can look up the rules in the application by pressing the rules button. They will then be presented by a page which explains the rules of the game in a short and easy way.

\begin{table}[h!]
\begin{tabular}{ | p{110pt} | p{250pt}  |}
\hline
\bf Source of stimulus &  End user  \\ \hline
\bf Stimulus & Learning system features \\ \hline 
\bf Artifact & System  \\  \hline
\bf Environment & At run-time \\ \hline
\bf Response & Proviedes a set of rules to play the game \\ \hline
\bf Response Measure & Reducing 30\% time of decision making \\ \hline

\end{tabular}
\caption{U1 - Learning game rules}
\end{table}

\pagebreak

\subsubsection{U2 - Show possible moves}

The player will continuously be presented with the possible actions he can make, possible actions can be removing the opponents pieces, moving your own pieces or placing new ones. There exist multiple rules for every action, and this will minimize confusion to players that are new or unfamiliar with the game. The possible actions will be highlighted with colors to make it as easy as possible for the users.

\begin{table}[h!]
\begin{tabular}{ | p{110pt} | p{250pt}  |}
\hline
\bf Source of stimulus &  End user  \\ \hline
\bf Stimulus & Minimize game confusion \\ \hline 
\bf Artifact & System  \\  \hline
\bf Environment & At run-time \\ \hline
\bf Response & Highlights possible moves \\ \hline
\bf Response Measure & Reducing 20\% time of decision making   \\ \hline

\end{tabular}
\caption{U2 - Show possible moves}
\end{table}

\subsection{Testability}

\subsubsection{T1 - Connection between the players}
The users should be able to find games that are awaiting players in a game lobby, they should from there get the necessary game information. The users could then join the game, and be greeted by the game with their respective names. 

\begin{table}[h!]
\begin{tabular}{ | p{110pt} | p{250pt}  |}
\hline
\bf Source of stimulus & System verifier  \\ \hline
\bf Stimulus & Testing connection \\ \hline 
\bf Artifact & System  \\  \hline
\bf Environment & At deployment time \\ \hline
\bf Response & Game welcomes both players to the lobby \\ \hline
\bf Response Measure & The connection should be completed within 10 seconds \\ \hline

\end{tabular}
\caption{T1 - Connection between the players}
\end{table}

\pagebreak



